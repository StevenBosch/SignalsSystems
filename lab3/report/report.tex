\documentclass{article}
\usepackage[
        a4paper,% other options: a3paper, a5paper, etc
        left=3cm,
        right=3cm,
        top=3cm,
        bottom=4cm,
        % use vmargin=2cm to make vertical margins equal to 2cm.
        % us  hmargin=3cm to make horizontal margins equal to 3cm.
        % use margin=3cm to make all margins  equal to 3cm.
]{geometry}
%\usepackage[utf8x]{inputenc}
\usepackage{graphicx}
\usepackage{caption}
\usepackage{enumerate}
\usepackage{subcaption}
\usepackage[procnames]{listings}
\usepackage{color}
\usepackage{amssymb}
\usepackage{amsmath}
\usepackage{comment}
\usepackage{hyperref}
\usepackage{blindtext}
\usepackage[titletoc,title]{appendix}
\usepackage{float}
\usepackage{fullpage}
\definecolor{codegreen}{rgb}{0,0.6,0}
\definecolor{codegray}{rgb}{0.5,0.5,0.5}
\definecolor{codepurple}{rgb}{0.58,0,0.82}
\definecolor{backcolour}{rgb}{0.95,0.95,0.92}

\lstdefinestyle{mystyle}{
    backgroundcolor=\color{backcolour},
    commentstyle=\color{codegreen},
    keywordstyle=\color{magenta},
    numberstyle=\tiny\color{codegray},
    stringstyle=\color{codepurple},
    basicstyle=\ttfamily,
    breakatwhitespace=false,
    breaklines=true,
    captionpos=t,
    keepspaces=true,
    numbers=left,
    numbersep=5pt,
    showspaces=false,
    showstringspaces=false,
    showtabs=false,
    tabsize=2
}
\lstset{style=mystyle, language=Matlab}
\renewcommand{\thesubsection}{\small(\alph{subsection})}

\title{Signals and Systems Lab 3}
\author{Maikel Withagen (s1867733) \and Steven Bosch (s1861948)}
\date{\today}

\begin{document}
\maketitle

\section{Convolution using FFTs}
Listing~\ref{myconv} gives our implementation of the convolution function using FFTs. To get the convolution of two signals using FFTs, we first pad both signals with the length of the other signal. After that we calculate the Fourier transform of both of the signals and multiply them. Finally we calculate the inverse Fourier transform of the acquired result and output its real part.

The following examples show that our implementation yields the same results as the built in convolution function of Octave:
\begin{lstlisting}
>>> myconv(1:5, 1:3)
ans = 1.0000    4.0000   10.0000   16.0000   22.0000   22.0000   15.0000
>>> conv(1:5, 1:3)
ans = 1    4   10   16   22   22   15

>>> myconv([0 1 0 -1], [-1 0 1 0])
ans = 0.00000  -1.00000  -0.00000   2.00000   0.00000  -1.00000  -0.00000

>>> conv([0 1 0 -1], [-1 0 1 0])
ans = 0  -1   0   2   0  -1   0
\end{lstlisting}
The results are equal, indicating that our implementation is correct.

 \lstinputlisting[caption={My convolution},label={myconv}]{../code/myconv.m}

\section{Discrete Fourier Transform}
Listings~\ref{VDM},~\ref{myDFT} and~\ref{myInvDFT} give our implementations of the Discrete Fourier Transform using the Vandermonde matrix, and its inverse. In $VDM.m$ the Van der Monde matrix is constructed, given an omega and a length of the signal N. Every element in the matrix holds the given omega to the power of the row index times the column index. In $myDFT.m$ the omega is calculated as $e^{2\pi i/N}$, with $N$ being the length of the signal $x$. It then calls on $VDM.m$ with this omega and length, and multiplies its result with the given signal $x$. The function $myInvDFT.m$ does almost the same thing, with the difference of omega being calculated as $e^{-2\pi i/N}$ and the final answer begin multiplied by $1/N$.

The following listing shows a comparison of these functions with the Octave functions $fft$ and $ifft$ ($myDFT$ gives $-0i$ instead of $+0i$, but this is of course equal):

\begin{lstlisting}
>>> myDFT(1:4)
ans = 10.0000 -  0.0000i   -2.0000 +  2.0000i   -2.0000 -  0.0000i   -2.0000 -  2.0000i

>>> fft(1:4)
ans = 10 +  0i   -2 +  2i   -2 +  0i   -2 -  2i

>>> myDFT([0 1 0 -1])
ans = 0.00000 - 0.00000i   0.00000 - 2.00000i   0.00000 + 0.00000i   -0.00000 + 2.00000i

>>> fft([0 1 0 -1])
ans = 0 + 0i   0 - 2i   0 + 0i   0 + 2i

>>> myInvDFT(1:4)
ans = 2.50000 - 0.00000i  -0.50000 - 0.50000i  -0.50000 + 0.00000i   -0.50000 + 0.50000i

>>> ifft(1:4)
ans = 2.50000 + 0.00000i  -0.50000 - 0.50000i  -0.50000 + 0.00000i   -0.50000 + 0.50000i

>>> myInvDFT([0 1 0 -1])
ans = 0.00000 - 0.00000i   0.00000 + 0.50000i   0.00000 - 0.00000i   -0.00000 - 0.50000i

>>> ifft([0 1 0 -1])
ans = 0.00000 + 0.00000i   0.00000 + 0.50000i   0.00000 + 0.00000i    0.00000 - 0.50000i
\end{lstlisting}
These examples show that the functions yield the same results.

 \lstinputlisting[caption={Van der Monde},label={VDM}]{../code/VDM.m}
 \lstinputlisting[caption={Discrete Fourier Transform},label={myDFT}]{../code/myDFT.m}
 \lstinputlisting[caption={Inverse Discrete Fourier Transform},label={myInvDFT}]{../code/myInvDFT.m}

\section{Number Theoretic Transform (following DFT)}
Listings~\ref{VDMprime},~\ref{NTT1} and~\ref{NTTinv1} give our implementations of
an adjusted Vandermonde matrix with the modulo of a given prime,
the Number Theoretic Transform and its inverse using the DFT steps.
In $VDMprime.m$ the same is done as in $VDM.m$,
except that after every multiplication a modulo has to be performed
in order to prevent the power's value to exceed the computational limits
Therefore this algorithm calculates every element of the matrix separately instead of per row like $VDM.m$ does.

$NTT1.m$ is very similar to $myDFT.m$.
 It also uses an omega, namely $omega = root^k$,
 where root is the primitive root of unity,
 calculated with the given function $rootsofunity.m$, and $k = (prime - 1)/N$.
 Finally it gives the product of the adjusted Vandermonde matrix and the signal x, modulo the prime.
 $NTTinv1.m$ does the same, except that it uses the modinverse of the root as omega,
and it multiplies the product of the Vandermonde matrix and the signals by the modinverse of N.

Since Octave does not have an NTT-function, it is hard to check our implementation.
What we can do is check whether using NTT and its inverse on some input yields
the input again. This is indeed the case, as we can see in the listing below:

\begin{lstlisting}
>>> NTTinv1(NTT1(4:-1:1))
ans = 4   3   2   1

>>> NTTinv1(NTT1([0 1 0 -1]))
ans = 0   1   0  -1
\end{lstlisting}

 \lstinputlisting[caption={Vandermonde matrix modulo prime},label={VDMprime}]{../code/VDMprime.m}
 \lstinputlisting[caption={NTT following the DFT algorithm},label={NTT1}]{../code/NTT1.m}
 \lstinputlisting[caption={Inverse NTT following the DFT algorithm},label={NTTinv1}]{../code/NTTinv1.m}

\section{Fast Fourier Transform}
Listings \ref{myFFT}, \ref{runFFT} and \ref{runInvFFT} give our implementations of the Fast Fourier Transform and its inverse. The function $myFFT.m$ is an implementation of the Cooley - Tukey algorithm, as is given in the slides. The `run'-functions create omega, given a signal $a$, and call on $myFFT$ with these parameters.

The following examples show the output of our FFT and IFFT functions and the Octave functions:

\begin{lstlisting}
>>> runFFT(1:4)
ans = 10.0000 +  0.0000i   -2.0000 +  2.0000i   -2.0000 +  0.0000i   -2.0000 -  2.0000i

>>> fft(1:4)
ans = 10 +  0i   -2 +  2i   -2 +  0i   -2 -  2i

>>> runFFT([0 1 0 -1])
ans = 0.00000 + 0.00000i   0.00000 - 2.00000i   0.00000 + 0.00000i   -0.00000 + 2.00000i

>>> fft([0 1 0 -1])
ans = 0 + 0i   0 - 2i   0 + 0i   0 + 2i
\end{lstlisting}

 \lstinputlisting[caption={FFT recursive function},label={myFFT}]{../code/myFFT.m}
 \lstinputlisting[caption={Run FFT},label={runFFT}]{../code/runFFT.m}
 \lstinputlisting[caption={Run inverse FFT},label={runInvFFT}]{../code/runInvFFT.m}

\section{Number Theoretic Transform (following FFT)}


\section{Convolution using NTTs}

\end{document}
