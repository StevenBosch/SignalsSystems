\documentclass{article}
\usepackage[
        a4paper,% other options: a3paper, a5paper, etc
        left=3cm,
        right=3cm,
        top=3cm,
        bottom=4cm,
        % use vmargin=2cm to make vertical margins equal to 2cm.
        % us  hmargin=3cm to make horizontal margins equal to 3cm.
        % use margin=3cm to make all margins  equal to 3cm.
]{geometry}
%\usepackage[utf8x]{inputenc}
\usepackage{graphicx}
\usepackage{caption}
\usepackage{enumerate}
\usepackage{subcaption}
\usepackage[procnames]{listings}
\usepackage{color}
\usepackage{amssymb}
\usepackage{amsmath}
\usepackage{comment}
\usepackage{hyperref}
\usepackage{blindtext}
\usepackage[titletoc,title]{appendix}
\usepackage{float}
\usepackage{fullpage}
\definecolor{codegreen}{rgb}{0,0.6,0}
\definecolor{codegray}{rgb}{0.5,0.5,0.5}
\definecolor{codepurple}{rgb}{0.58,0,0.82}
\definecolor{backcolour}{rgb}{0.95,0.95,0.92}

\lstdefinestyle{mystyle}{
    backgroundcolor=\color{backcolour},
    commentstyle=\color{codegreen},
    keywordstyle=\color{magenta},
    numberstyle=\tiny\color{codegray},
    stringstyle=\color{codepurple},
    basicstyle=\ttfamily,
    breakatwhitespace=false,
    breaklines=true,
    captionpos=t,
    keepspaces=true,
    numbers=left,
    numbersep=5pt,
    showspaces=false,
    showstringspaces=false,
    showtabs=false,
    tabsize=2
}

\lstset{style=mystyle, language=Matlab}
\renewcommand{\thesubsection}{\small(\alph{subsection})}

\title{Signals and Systems Lab 3}
\author{Maikel Withagen (s1867733) \and Steven Bosch (s1861948)}
\date{\today}

\begin{document}
\maketitle

\section{Convolution using FFTs}
Listing \ref{myconv} gives our implementation of the convolution function using FFTs. To get the convolution of two signals using FFTs, we first pad both signals with the length of the other signal. After that we calculate the Fourier transform of both of the signals and multiply them. Finally we calculate the inverse Fourier transform of the acquired result and output its real part.

The following examples show that our implementation yields the same results as the built in convolution function of Octave:
\begin{lstlisting}
>>> myconv(1:5, 1:3)
ans = 1.0000    4.0000   10.0000   16.0000   22.0000   22.0000   15.0000
>>> conv(1:5, 1:3)
ans = 1    4   10   16   22   22   15

>>> myconv([0 1 0 -1], [-1 0 1 0])
ans = 0.00000  -1.00000  -0.00000   2.00000   0.00000  -1.00000  -0.00000

>>> conv([0 1 0 -1], [-1 0 1 0])
ans = 0  -1   0   2   0  -1   0
\end{lstlisting}

 \lstinputlisting[caption={My convolution},label={myconv}]{../code/myconv.m}
 
\section{Discrete Fourier Transform}
Listings \ref{VDM}, myDFT, myInvDFT} give our implementation of the Discrete Fourier Transform using the Vandermonde matrix, and its inverse. 
 \lstinputlisting[caption={Van der Monde},label={VDM}]{../code/VDM.m}
 \lstinputlisting[caption={Discrete Fourier Transform},label={myDFT}]{../code/myDFT.m}
 \lstinputlisting[caption={Inverse Discrete Fourier Transform},label={myInvDFT}]{../code/myInvDFT.m}
 
\section{Number Theoretic Transform (following DFT)}

 \lstinputlisting[caption={NTT following the DFT algorithm},label={NTT1}]{../code/NTT1.m}
 \lstinputlisting[caption={Inverse NTT following the DFT algorithm},label={NTTinv1}]{../code/NTTinv1.m}
 
\section{Fast Fourier Transform}
 \lstinputlisting[caption={FFT recursive function},label={myFFT}]{../code/myFFT.m}
 \lstinputlisting[caption={Run FFT},label={runFFT}]{../code/runFFT.m}
 \lstinputlisting[caption={Run inverse FFT},label={runInvFFT}]{../code/runInvFFT.m}

\section{Number Theoretic Transform (following FFT)}

\section{Convolution using NTTs}
 
\end{document}
